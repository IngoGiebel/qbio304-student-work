\section{Abstract}

Rice is the most widely consumed staple food for a large part of the world's human population, and drought is the most imperative and major limitation for rice production in rainfed ecosystems. Therefore, there is a great requirement of rice varieties with drought tolerance, and much research is done to improve the drought tolerance of rice. The understanding of drought stress responses on the gene level may substantially contribute to this goal.

Here, the impact of drought stress on the gene expression of wild and cultivated rice is examined with the following statistical methods: hierarchical clustering analysis (HCA)\index{hierarchical cluster analysis (HCA)}, principal component analysis (PCA)\index{principal component analysis (PCA)}, functional enrichment analysis\index{functional enrichment analysis}. The analyzed RNA-seq\index{RNA-seq} data was publicly submitted by the Institute of Botany, Chinese Academy of Sciences, in January 2021.

This analysis reveals that both the wild and the cultivated rice species react strongly to drought stress by down- and up-regulating their gene activity. The hierarchical cluster analysis\index{hierarchical cluster analysis (HCA)} of the two different wild rice cultivars further uncovers that the examined cultivar is an important confounding factor. When considering the ontology of the down- and up-regulated genes, both rice species largely coincide with respect to the significantly enriched GO terms. Another result is that on the gene level the impact of drought stress may be best described and categorized in terms of biological processes (BP)\index{gene ontology (GO)!biological process (BP)} as opposed to the molecular functions (MF)\index{gene ontology (GO)!molecular function (MF)} of the genes and as opposed to the cellular components (CC)\index{gene ontology (GO)!cellular component (CC)} in which the gene products are physically located.