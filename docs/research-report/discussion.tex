\section{Discussion}

This analysis reveals that both species \plant[a]{On} and \plant[a]{Os} react strongly to drought stress by down- and up-regulating their gene activity. According to the limma tests (p-value = 1 \%), the number of significantly down- and up-regulated genes is much higher for the wild rice than for the cultivated rice: 48 genes in the wild rice vs 18 genes in the cultivated rice. These results are consistent with current research results: Drought tolerance is regulated by several genes \autocite{10.3389/fpls.2016.01029}. Furthermore, wild and cultivated species of rice have distinctive proteomic responses to drought stress \autocite{10.3390/ijms21175980}.

The hierarchical cluster analysis\index{hierarchical cluster analysis (HCA)} of the two \plant[a]{On} cultivars further uncovers that the examined cultivar might have an even greater impact on the gene expressions than drought stress conditions. However, when considering the ontology of the down- and up-regulated genes, both rice species largely coincide with respect to the significantly enriched GO terms. Another result is that on the gene level the impact of drought stress may be best described and categorized in terms of biological processes (BP)\index{gene ontology (GO)!biological process (BP)} as opposed to the molecular functions (MF)\index{gene ontology (GO)!molecular function (MF)} of the genes and as opposed to the cellular components (CC)\index{gene ontology (GO)!cellular component (CC)} in which the gene products are physically located.

The initial quality assessment indicate a good quality of the examined RNA-seq data. However, the principal component analysis\index{principal component analysis (PCA)} indicates that one of the three \plant[a]{Os} samples might have a batch effect\index{batch effect}. Irrespective of this, the general results of this analysis should be well reproducible.

Rice is the most widely consumed staple food for a large part of the world's human population, and drought is the most imperative and major limitation for rice production in rainfed ecosystems. Therefore, there is a great requirement of rice varieties with drought tolerance, and much research is done to improve the drought tolerance of rice \autocite{10.1016/j.rsci.2021.01.002}.

This study analyses leaf tissues from seedlings at the age of twelve days. For a better understanding of the adaptation mechanisms to drought stress conditions, it would be necessary to also examine and compare other rice species and cultivars, especially drought-tolerant versus drought-sensitive cultivars. Furthermore, different tissues and ages should be examined. The function of the differentially expressed genes need to be analyzed in detail.

The research of drought stress responses on the gene level may substantially contribute to the breeding for drought tolerant rice varieties.