\section{Introduction}

RNA-sequencing is used to analyze the transcriptome\index{transcriptome}, indicating which of the genes encoded in the DNA are turned on or off and to what extent. This study examines the impact of drought stress on the gene expression of wild and and cultivated rice, using statistical methods. To this end, publicly available RNA-seq\index{RNA-seq} data from the Institute of Botany, Chinese Academy of Sciences, submitted on January 1st, 2021, is analyzed with respect to differential gene expression. This data allows for a direct comparison of normal and drought stress conditions for two closely related rice species: wild rice (Oryza nivara\index{Oryza!nivara}), and cultivated rice (Oryza sativa\index{Oryza!sativa}). Furthermore, it allows for the comparison of the two wild rice cultivars BJ278{cultivar!BJ278} and BJ89{cultivar!BJ89}.